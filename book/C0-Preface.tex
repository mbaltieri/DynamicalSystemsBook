\documentclass[DynamicalBook]{subfiles}
\begin{document}
%


\setcounter{chapter}{0}%Just finished 0.


%------------ Chapter ------------%
\chapter{Preface}\label{chapter.0}

This book is a work in progress --- including the acknowledgements below! Use at your own peril!

Categorical systems theory is an emerging field of mathematics which seeks to apply the methods of category theory to general systems theory. General systems theory is the study of systems --- ways things can be and change, and models thereof --- in full generality. The difficulty is that there doesn't seem to be a single core idea of what it means to be a ``system''. Different people have, for different purposes, come up with a vast array of different modeling techniques and definitions that could be called ``systems''. There is often little the same in the precise content of these definitions, though there are still strong, if informal, analogies to be made accross these different fields. This makes coming up with a mathematical theory of general systems tantalizing but difficult: what, after all, is a system in general?

Category theory has been describe as the mathematics of formal analogy making. It allows us to make analogies between fields by focusing not on content of the objects of those fields, but by the ways that the objects of those fields relate to one another. Categorical systems theory applies this idea to general systems theory, avoiding the issue of not having a contentful definition of system by instead focusing on the ways that systems interact with eachother and their environment.

These are the main ideas of categorical systems theory:
\begin{enumerate}
\item Any system interacts with its environment through an \emph{interface}, which can be described separately from the system itself.
\item All interactions of a system with its environment take place through its interface, so that from the point of view of the environment, all we need to know about a system is what is going on at the interface.
\item Systems interact with other systems through their respective interfaces. So, to understand complex systems in terms of their component subsystems, we need to understand the ways that interfaces can be connected. We call these ways that interfaces can be connected \emph{composition patterns}.
\item Given a composition pattern describing how some interface are to be connected, and some systems with those interfaces, we should have a \emph{composite} system which consists of those subsystems interacting according to the composition pattern. The ability to form composite systems of interacting component systems is called \emph{modularity}, and is a well known boon in the design of complex systems.
\end{enumerate}

In a sense, the definitions of categorical systems theory are all about \emph{modularity} --- how can systems be composed of subsystems. On the other hand, the theorems of categorical systems theory often take the form of \emph{compositionality} results. These say that certain facts and features of composite systems can be understood or calculated in terms of their component systems and the composition pattern.

This book will follow this general paradigm. We will see definitions of systems which foreground modularity --- the ways that systems can be composed to form more complex systems. And then we will prove a general compositionality theorem, showing that a large class of behaviors of composite systems can be calculated in terms of their components and the composition pattern.

This abstract overview leaves a lot of questions to be answered. What is, or what can be a system? What is an interface? What is a composition pattern? How do we compose systems using composition patterns? What is a behavior of a system, and how do we study it categorically? There is no single answer to this suite of questions. Different people working with different aims will answer these questions differently. But we can package this suite of questions into an informal definition of a \emph{doctrine} of dynamical systems.

\begin{informal}\label{informal:paradigm}
 A \emph{doctrine of dynamical systems} is a particular way to answer the following questions about it means to be a \emph{systems theory}:
 \begin{itemize}
   \item What does it mean to be a system? Does it have a notion of states, or of behaviors? Or is it a diagram describing the way some primitive parts are organized?
   \item What should the interface of a system be?
   \item How can interfaces be connected in composition patterns?
  \item How are systems composed through composition patterns between their interfaces.
  \item What is a map between systems, and how does it affect their interfaces?
  \item When can maps between systems be composed along the same composition patterns as the systems.
  \end{itemize}
  \end{informal}

  We will give a semi-formal\footnote{And for experts, a formal definition, though we won't fully justify it.} definition of dynamical systems doctrine in \cref{chapter.6}. For the first five chapters of this book on the other hand, we will work within a fixed doctrine of dynamical systems which we might call the \emph{parameter-setting} doctrine. This doctrine gives a particular answer to the above questions, based around the following defintion of a \emph{system}.

\begin{informal}\label{inf.dynam_sys}
  A \emph{dynamical system} consists of:
  \begin{itemize}
  \item a notion of how things can be, called the \emph{states}, and
  \item a notion of how things will change given how they are, called the \emph{dynamics}.
  \end{itemize}
  The dynamics of a system might also depend on some free \emph{parameters} or \emph{inputs} that are imported from the environment, and
  we will often be interested in some particular \emph{variables} of the
  state that are \emph{exposed} or \emph{output} to the environment.
\end{informal}

In the first two chatpers, we will see a variety of examples of such systems, including discrete-time deterministic systems, systems of differential equations, and non-deterministic systems such as Markov decision processes. We will also see what composition patterns can be in the parameter-setting doctrine; they can be drawn as wiring diagrams like this:
\[
\begin{tikzpicture}[oriented WD, every fit/.style={inner xsep=\bbx, inner ysep=\bby}, bb small]
  \node[bb={2}{2}, fill=blue!10] (X1) {};
  \node[bb={1}{1}, fit={(X1) ($(X1.north)+(0,1)$)}, densely dotted] (Y1) {};
  \node[bb={2}{1}, fill=blue!10, below right=4 and 0 of X1] (X2) {};
  \node[bb={0}{2}, fill=blue!10, below left=of X2] (X3) {};
  \node[bb={1}{2}, fit=(X2) (X3), densely dotted] (Y2) {};
  \node[bb={1}{2}, fit=(Y1) (Y2)] (Z) {};
  \draw[ar] (Z_in1') to (Y2_in1);
  \draw[ar] (Y2_in1') to (X2_in1);
  \draw[ar] (X3_out1) to (X2_in2);
  \draw[ar] (X3_out2) to (Y2_out2');
  \draw (Y2_out2) to (Z_out2');
  \draw[ar] (Y1_in1') to (X1_in2);
  \draw (X1_out2) to (Y1_out1');
  \draw[ar] (Y1_out1) to (Z_out1');
  \draw (X2_out1) to (Y2_out1');
  \draw[ar] let \p1=(Y2.north east), \p2=(Y1.south west), \n1={\y1+\bby}, \n2=\bbportlen in
          (Y2_out1) to[in=0] (\x1+\n2,\n1) -- (\x2-\n2,\n1) to[out=180] (Y1_in1);
  \draw[ar] let \p1=(X1.north east), \p2=(X1.north west), \n1={\y1+\bby}, \n2=\bbportlen in
          (X1_out1) to[in=0] (\x1+\n2,\n1) -- (\x2-\n2,\n1) to[out=180] (X1_in1);
\end{tikzpicture}
\]

But \cref{inf.dynam_sys} is not so precise. Deterministic systems, systems of differential equations, Markov decision processes, and many more sorts of systems fit the mold, but they also differ in many important ways. \cref{inf.dynam_sys} doesn't tell us what the states should be (a set? a topological space? a manifold? a graph? something else?), and it doesn't tell us what it means to specify how things change given how they are. We can package \emph{this} suite of questions into the notion of a \emph{theory of dynamical systems}, or \emph{systems theory} for short.

\begin{informal}\label{informal.doctrine}
  A \emph{theory} of dynamical systems --- or a \emph{systems theory} for short --- is a particular way to answer the following
  questions about what it means to be a dynamical system:
  \begin{itemize}
  \item What does it mean to be a state?
  \item How should the output vary with the state --- discretely,
    continuously, linearly?
  \item Can the kinds of input a
    system takes in depend on what it's putting out, and how do they depend on it?
  \item What sorts of changes are possible in a given state?
  \item What does it mean for states to change.
  \item How should the way the state changes vary with the input?
  \end{itemize}
\end{informal}

We will make this definition fully formal in \cref{Chapter.3}, after introducing enough category theory to state it. Once we have made the definition of systems theory formal, we can make the definition of system. But what is interesting about dynamical systems is how they \emph{behave}.
\begin{informal}\label{inf:behavior}
  A \emph{behavior} of a dynamical system is a particular way its states can
  change according to its dynamics.

  There are different \emph{kinds of behavior} corresponding to the different
  sorts of ways that the states of a system could evolve. Perhaps they eventually
  repeat, or they stay the same despite changing conditions.
\end{informal}

In \cref{Chapter.3}, we will formalize this definition of behavior for each systems theory by noticing that for any given kind of behavior, there is almost always a system that \emph{represents} that behavior, in that it does exactly that behavior and nothing more. For example, a point moving uniformly on a line represents a trajectory, and a point moving on a circle represents a periodic orbit. We will also note that a particular behavior of a system will alway requires a particular choice of parameters, which we call the \emph{chart} of the behavior.

Using this realization, we will prove our main compositionality theorem in \cref{Chapter.5}. This theorem states, informally, the following facts concerning the composition of systems.
\begin{itemize}
  \item Suppose that we are wiring
    our systems together in two stages. If we take a
    bunch of behaviors whose charts are compatible for the total wiring pattern
    and wire them together into a behavior of the whole system, this is the same
    behavior we get if we first noticed that they were compatible for the first
    wiring pattern, wired them together, then noticed that the result was
    compatible for the second wiring pattern, and wired that together. This
    means that nesting of wiring diagrams commutes with finding behaviors of our systems.
  \item Suppose that we have two
    charts and a behavior of each. Then composing a behavior with the composite of those behaviors is the same as composing
    it with the first one and then with the second one.
  \item Suppose that we have a pair of
    wiring patterns and compatible charts between them. If we
    take a bunch of behaviors whose charts are compatable according to the first
    wiring pattern, wire them together, and then compose with a behavior of the
    second chart, we get the same thing as if we compose them all with behaviors
    of the first chart, noted that they were compatible with the second wiring
    pattern, and then wired them together.
\end{itemize}

These basic principles show us how the problem of understanding the behaviors of composite systems can be broken down consistently into the hopefully smaller problems of understanding the behaviors of their components, and the pattern of composition.

This theorem comes down to some fully abstract category theory: the construction of \emph{representable lax doubly indexed functors}. Since the theorem is abstract, it can be applied not only to any systems theory as in \cref{informal.doctrine}, but any systems theory in any \emph{doctrine} (\cref{informal:paradigm}). In \cref{chapter.6}, we will see two other doctrines which give us substantially different ways to think about systems theory. But the compositionality theorem proven in \cref{Chapter.5} will apply to them as well.

This book is intended as a first guide to the rapidly growing field of categorical systems theory. While the book does presume a knowledge of basic category theory (which can be gained from any one of the many wonderful introductions to the subject --- see \cref{sec:category.theory}), the special topics needed for the definitions and theorems --- indexed categories, double categories, doubly indexed categories and their functors --- will be introduced as they become necessary.

My hope is that this book can inspire you to use categorical methods in systems theory in your work, whenever they are useful, and to demand more from these tools where they are not yet useful.







\section*{Acknowledgments}

David Spivak has been a friend and mentor to me as I write this book and beyond.
In many ways, I see this book as my take on David's research in lens based
systems in recent years. David and I began writing a book together, of which
this book was to be the first half and David's book on polynomial functors (now
co-authored with Nelson Niu) was to be the second. But as we were writing, we
realized that these weren't two halves of the same book, but rather two books
in the same genre. It was a great pleasure writing with David during the summer
of 2020, and I owe him endless thanks for ideas, suggestions, and great
conversation. This book wouldn't exist without him.

Emily Riehl has been a better advisor than I could have thought to have asked for. I want to thank her for her financial support (through grant ????) during the development of much of the mathematics in this book. I'll write more in my thesis, but as far as this book goes, I would like to thank her for her careful reading, her advice on logistics, and her patience.

Thanks go to Emily Riehl, tslil clingman, Sophie Libkind, John Baez, Geoff Cruttwell, Brendan
Fong, Christian Williams.

Thanks to Henry Story for pointing out typos.

This book was written with support from the Topos Institute.


\end{document}
